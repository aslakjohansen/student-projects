%%%%%%%%%%%%%%%%%%%%%%%%%%%%%%%%%%%%%%%%%%%%%%%%%%%%%%%%%%%%%%%
%%%%%%%%%%%%%%%%%%%%%%%%%%%%%%%%%%%%%%%%%%%%%%%%%%%%%%%% Themes
\section{Introduction}
\begin{frame}
    \frametitle{Introduction}
    \vspace{3mm}
    These slides form an overview of those themes and domains that best define my primary interests. However, I am keen to find new domains that can benefit from similar approaches, and new approaches the can benefit my existing domains of interest.
    
    \vspace{3mm}
    In general:
    \begin{itemize}
      \item Software-define metadata.
      \item In-network processing.
      \item Designing for polling-free/push-based operations.
    \end{itemize}
    
    \vspace{3mm}
    Technology stack: C, Rust, Python, Elixir, Go, Nerves, Phoenix, Neo4J, Cypher, TimescaleDB, Cairo, SVG, GIT, FreeRTOS \ldots
    
    \vspace{3mm}
    More project ideas are available here: \url{https://github.com/aslakjohansen/student-projects}
\end{frame}

%%%%%%%%%%%%%%%%%%%%%%%%%%%%%%%%%%%%%%%%%%%%%%%%%%%%%%%%%%%%%%%
%%%%%%%%%%%%%%%%%%%%%%%%%%%%%%%%%%%%%%%%%%%%%%%%%%%%%%%% Themes
\section{Themes}
\begin{frame}
    \vspace{25mm}
    \begin{center}
        \Huge{Part 1:\\Themes}
    \end{center}
\end{frame}

\subsection{Concurrency}
\begin{frame}
    \frametitle{Concurrency}
    \vspace{0mm}
    Logic defined in a way that allows for parallelism but doesn't require it.
    \vspace{3mm}
    \begin{itemize}
        \item Concurrency models:
          \begin{itemize}
            \item Granularity: From threads to CSP.
            \item Actor Model (e.g., as used in Elixir).
          \end{itemize}
        \item Introspection of concurrent systems.
    \end{itemize}
\end{frame}

\subsection{Availability}
\begin{frame}
    \frametitle{Availability}
    \vspace{0mm}
    Keeping a service available despite faults in soft- and hardware. 
    \vspace{3mm}
    \begin{itemize}
        \item Fault tolerance.
        \item Graceful degredation.
        \item Supervision trees.
        \item Netsplits and resulting split brain scenarios.
        \item Hot-updates.
        \item Live introspection.
    \end{itemize}
\end{frame}

\subsection{Responsiveness}
\begin{frame}
    \frametitle{Responsiveness}
    \vspace{0mm}
    Keeping the system responsive despite a high load.
    \vspace{3mm}
    \begin{itemize}
        \item Bounding concurrency through back pressure.
        \item Push-based data flow transformes problems to routing problems.
    \end{itemize}
\end{frame}

\subsection{Metadata}
\begin{frame}
    \frametitle{Metadata}
    \vspace{0mm}
    Modeling and interaction with data used to describe other data.
    \vspace{2mm}
    \begin{itemize}
        \item Software-defined metadata where the decision of \textsl{how} parts of a query against a metadata model are resolved can be taken dynamically, and be implemented by arbitrary logic. This can take physical state into account or extend the set of available physical sensors (from the model) with virtual sensors that are spun up on demand.
        \item Interactions models for metadata.
          \begin{itemize}
            \item Query-based subscriptions.
          \end{itemize}
        \item Metadata lifecycle.
          \begin{itemize}
            \item Metadata as a timeseries.
            \item Correction to historical metadata.
          \end{itemize}
        \item Graphs.
        \item Ontologies.
        \item Papertrail of the processing steps that led to some data being produced.
    \end{itemize}
\end{frame}

\subsection{Efficiency}
\begin{frame}
    \frametitle{Efficiency}
    \vspace{0mm}
    Software that operates at \textsl{reasonable} levels of efficiency.
    \vspace{3mm}
    \begin{itemize}
        \item Green computing.
        \item API design for efficient operations.
        \item Caching-Friendly DAG-Based Data Processing
    \end{itemize}
\end{frame}

\subsection{Automation}
\begin{frame}
    \frametitle{Automation}
    \vspace{0mm}
    Moving tasks to software to enable repeated processing at low cost.
    \vspace{3mm}
    \begin{itemize}
        \item Manual-first principle.
        \item Tooling for manipulating SVG figures:
          \url{https://github.com/aslakjohansen/svg-narrative}
    \end{itemize}
\end{frame}

%%%%%%%%%%%%%%%%%%%%%%%%%%%%%%%%%%%%%%%%%%%%%%%%%%%%%%%%%%%%%%%
%%%%%%%%%%%%%%%%%%%%%%%%%%%%%%%%%%%%%%%%%%%%%%%%%%%%%%% Domains
\section{Domains}
\begin{frame}
    \vspace{25mm}
    \begin{center}
        \Huge{Part 2:\\Domains}
    \end{center}
\end{frame}

\subsection{Buildings}
\begin{frame}
    \frametitle{Buildings}
    \vspace{3mm}
    \begin{itemize}
        \item Buildings are cypher-physical systems.
        \item Instrumented using a mixture of:
          \begin{itemize}
            \item Old-school wired automation systems.
            \item Ad-hoc modern IoT installations.
          \end{itemize}
        \item Logic is typically hardcoded instead of being informed by an exposed metadata model.
        \item This makes it very costly to (i) make adjustments to existing systems, and (ii) introduce new software systems.
        \item And that, in turn, makes it expensive to transform data from existing sensors to actionable insights.
    \end{itemize}
\end{frame}

\subsection{Environment}
\begin{frame}
    \frametitle{Environment}
    \vspace{3mm}
    Themes:
    \begin{itemize}
        \item Monitoring of environmental properties using battery- or solar-powered sensor boards.
        \item Processing of multispectral satellite images (e.g., to count trees and create a heatmap of average tree diameter).
        \item Using a camera as a sensor.
        \item Robust LoRa-based data collection network for environmental data.
        \item The EU taxonomy for sustainable activities:
          \\
          \scalebox{0.5}{\url{https://ec.europa.eu/info/business-economy-euro/banking-and-finance/sustainable-finance/eu-taxonomy-sustainable-activities_en}}
    \end{itemize}
\end{frame}

\subsection{Society}
\begin{frame}
    \frametitle{Society}
    \vspace{3mm}
    Themes:
    \begin{itemize}
        \item Climate change mitigation and adaptation
        \item Privacy under comfort
          \begin{itemize}
            \item Automatic fillout of cookie preferences
          \end{itemize}
        \item Getting trustworthy, verifiable and retractable references into the news cycle
        \item General open source tool support (e.g., for teaching; \#not-selfish). Examples:
          \begin{itemize}
            \item Flexible clicker platform
              \\
              \scalebox{.8}{\url{https://github.com/aslakjohansen/student-projects/blob/master/db/clicker-system.md}}
            \item Privacy Analyzer
              \\
              \scalebox{.8}{\url{https://github.com/aslakjohansen/student-projects/blob/master/db/privacy-analyzer.md}}
            \item Practical cookie agreement analysis
              \\
              \scalebox{.8}{\url{https://github.com/aslakjohansen/student-projects/blob/master/db/cookie_agreements.md}}
          \end{itemize}
    \end{itemize}
\end{frame}

%%%%%%%%%%%%%%%%%%%%%%%%%%%%%%%%%%%%%%%%%%%%%%%%%%%%%%%%%%%%%%%
%%%%%%%%%%%%%%%%%%%%%%%%%%%%%%%%%%%%%% Sample Project Selection
\section{Sample Project Selection}
\begin{frame}
    \vspace{25mm}
    \begin{center}
        \Huge{Part 3:\\Sample Project Selection}
    \end{center}
\end{frame}

\subsection{PubSub Broker Supporting Demand-Driven Publications}
\begin{frame}
    \frametitle{PubSub Broker Supporting Demand-Driven Publications}
    \vspace{0mm}
    The Publish-Subscribe pattern allows subscribers to receive data from publishers via named channels. A broker (or more) are used to match up publications to subscribers, but also serves the role of decoupling the two sides.
    
    \vspace{3mm}
    Accordingly, a subscriber does not know is anyone is consuming the produced data, and thus the pattern provides no means of automatically deciding wheter a datastream should be produced or not.
    
    \vspace{3mm}
    This project seeks to add a step whereby the publishers -- through a form of two-way pubsub -- are informed of the ongoing subscriptions. Based on this they can then decide on which datastreams to publish. The improved efficiency model will allow for a larger suite of datastreams to be offered, while only \textsl{paying} for those being consumed.

%    \vspace{3mm}
%    Full text: \scalebox{0.85}{\url{https://github.com/aslakjohansen/student-projects/blob/master/db/?.md}}
\end{frame}

\subsection{Cypher to SparQL Adaptor Service}
\begin{frame}
    \frametitle{Cypher to SparQL Adaptor Service}
    \vspace{0mm}
    RDF is a language for describing ontologies using triples. OWL builds a class model on top of RDF. SparQL allows us to query this. Due to the level of abstraction such models have a number of benefits (e.g., validation and reasoning) but also a learning curve that is prohibitive.
    
    \vspace{3mm}
    Cypher, on the other hand, is a query language for property graphs. It is easy to reason about, and operates at at level of abstraction that is comparable to OWL.
    
    \vspace{3mm}
    This project seeks to construct a server that exposes a service which takes a Cypher query as input, parses it, transforms it to SparQL (that targets an OWL level of abstraction) and uses another service to resolve it against an OWL model.
%    \vspace{3mm}
%    Full text: \scalebox{0.85}{\url{https://github.com/aslakjohansen/student-projects/blob/master/db/?.md}}
\end{frame}

\subsection{D3 Phoenix LiveView Components}
\begin{frame}
    \frametitle{D3 Phoenix LiveView Components}
    \vspace{0mm}
    Phoenix is a modern web framework that allows for quick prototyping of web applications while keeping a simple path to transitioning it to production. It is build around a MVC component model called LiveView that crosses the client/server boundary and naturally provides fast response times and hot updates. As long as existing components are applicable, not JavaScript code is necessary to make a website. The set of LiveView components for plotting is limited.
    
    \vspace{3mm}
    D3 is an attractive JavaScript framework for plotting.
    
    \vspace{3mm}
    The goal for this project is to:
    \begin{enumerate}
      \item Produce a number of both common and representative usecases.
      \item Generalize these to increase applicability.
      \item Produce a number of LiveView components that supports these generalized usecases.
      \item Document the process of producing LiveView components for D3.
    \end{enumerate}

%    \vspace{3mm}
%    Full text: \scalebox{0.85}{\url{https://github.com/aslakjohansen/student-projects/blob/master/db/?.md}}
\end{frame}

\subsection{Make Media Credible Again}
\begin{frame}
    \frametitle{Make Media Credible Again}
    \vspace{0mm}
    During the last decade or so, the credibility of online media has taken a dive. It has gotten so bad that there is no longer a shared sense of reality, and that has severe democratic implications.
    
    \vspace{3mm}
    We would like to create a system that allows us to track the premise graph of news:
    \begin{itemize}
      \item Peer review based on user preferences rooted in a social network.
      \item Users will see credibility scores based on functions over their own set of trusted references. This makes the score subjective, and that subjectiveness should be observable through cluster and fragility analysis.
      \item Cycle detection to highlight that an "argument" is cyclic in nature and thus invalid.
      \item Subgraph invalidation when a premise is debunked.
      \item Subscription to invalidation. So that users get notified whenever a document that they have consumed is found invalid.
    \end{itemize}

    \vspace{3mm}
    Full text: \scalebox{0.85}{\url{https://github.com/aslakjohansen/student-projects/blob/master/db/mmca.md}}
\end{frame}

